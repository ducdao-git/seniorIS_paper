\section{YOLOv5}  \label{sec:yolov5}

The YOLOv5 is the fifth entry of the YOLO family. The model is published in 2020 by Ultralytics teams \cite{yolov5_github}. The name YOLOv5 is still controlversal to this date because it is considered as less innovative compare to the YOLOv4 model. While YOLOv4 has a significant change in structure and use some state-of-art algorithm like MISH activation function and GIOU(Generalized Intersection over Union) loss function, YOLOv5 is more focus on ease of use, model size control, and enhancing training data \cite{yolov5_review}. Unfortunately, YOLOv5 model have never had a formal research paper detail explaining the implementation detail, but it has a well documented and supported API developed and maintain by the Ultralytics teams \cite{yolov5_github}. 

At it core, YOLOv5 is YOLOv3 structure with flexible control of model size \cite{yolov5_review}. This is reflected by the API where there are 5 versions of YOLOv5: nano, small, medium, large, and xlarge. Where nano is the smallest version with approximate 12.7 million parameters and xlarge is the largest version with approximate 141.8 million parameters \cite{pytorch_yolov5}. With the different in size, without a doubt, nano version is much faster than the xlarge version at 4.3ms compare to 22.4ms. However, nano version also have a smaller mAP score compare to the xlarge version at 61.9\% compare to 72.0\% on the COCO evaluation set at IoU threshold of 0.5.