\section{YOLOv5}  \label{sec:yolov5}

The YOLOv5 is the fifth entry of the YOLO family. The model was published in 2020 by Ultralytics teams \cite{yolov5_github}. The name YOLOv5 is still controversial to this date because it is considered less innovative compared to the YOLOv4 model \cite{yolov5_review}. While YOLOv4 has a significant change in structure and use some state-of-art algorithm like the MISH activation function and GIOU(Generalized Intersection over Union) loss function \cite{yolov4_2020}, YOLOv5 is more focused on ease of use, model size control, and enhancing training data \cite{yolov5_review}. Unfortunately, the YOLOv5 model has never had a formal research paper detail explaining the implementation detail, but it has a well-documented and supported API developed and maintained by the Ultralytics teams \cite{yolov5_github}. 

At its core, YOLOv5 is a YOLOv3 structure with flexible control of model size \cite{yolov5_review}. This is reflected by the API, where there are five versions of YOLOv5: nano, small, medium, large, and xlarge  \cite{pytorch_yolov5}. Where nano is the smallest version with approximately 12.7 million parameters, and xlarge is the largest version with approximately 141.8 million parameters. With the difference in size, without a doubt, the nano version is much faster than the xlarge version at 4.3ms compared to 22.4ms. However, the nano version also has a smaller mAP score compared to the xlarge version, 61.9\% compared to 72.0\% on the COCO evaluation set at the IoU threshold of 0.5.