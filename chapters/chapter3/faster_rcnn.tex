\section{Faster R-CNN} \label{sec:faster_rcnn}

% Faster R-CNN is an upgraded version of the original R-CNN (Regional Convolutional Neural Network). It was developed by Ren et al. in 2015 for object detection tasks and has since become one of the most popular and widely used deep learning models for this purpose, due to its effectiveness and efficient speed.

% Rather than generating individual regions as in R-CNN, Faster R-CNN uses a single convolutional network to generate what's known as a region proposal network (RPN). The network takes an input image, divides it into several regions, and then applies classificational algorithms to each region to detect objects. This method has proven to be much faster than the original R-CNN model.

% The result of this approach is that it can process images at a faster speed while also boosting accuracy. In comparison to other popular detection algorithms such as YOLOv4 (You Only Look Once), Faster R-CNN demonstrates superior performance for more complex tasks such as small-object detection and large object recognition with overlapping elements. Furthermore, when combined with data augmentation techniques like transfer learning or pretrained models like ResNet-50, Faster R-CNN can achieve near real-time speeds while still maintaining high accuracy rates of up to 70%.

% In addition to being fast and accurate, Faster R-CNN also offers scalability which enables users to apply it on multiple GPUs or TPUs if needed. This makes it suitable for larger datasets and projects that demand higher throughput rates or require more computing power. Finally, due to its modular structure, users can also easily modify certain parts of the model without having to retrain the entire model from scratch making it extremely user friendly and customizable without sacrificing performance or accuracy.