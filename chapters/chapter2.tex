%!TEX root = ../username.tex
\chapter{Instance Segmentation} \label{chap:instance_segmentation}

An instance segmentation task is a computer vision task that involves
determining the boundaries and identities of individual objects within an image
or video. Unlike semantic segmentation, which classifies each pixel in an image
as belonging to a particular class, instance segmentation assigns a unique
label to each object in a scene and then identifies its boundaries. For
example, in a picture of two vehicles one after another on the street, instance
segmentation would be able to distinguish between the two vehicles by assigning
them different labels and draw bounding box around them. 

The main distinction between instance segmentation and semantic segmentation
lies in the level of detail they provide. While semantic segmentation will
divide an image into classes such as "vehicle" or "animal", instance
segmentation will offer more details by distinguishing between each individual
object within those classes. It does this by assigning a unique label to each
object for identification. This means that with instance segmentation, it's
possible to identify not only what type of thing is present within an image but
also where it is located and how many there are - something semantic
segmentation cannot do on its own. 

On the other hand, instance segmentation and object dectection are quite
similar on a high level. Upon comparing instance segmentation with object
detection tasks reveals some further distinctions between the two tasks. While
both involve locating objects within an image or video stream, object detection
focuses on providing information about the location of the detected objects
while instance segmentation goes further by identifying their outlines as well
as providing additional information about them like the object size.
Additionally, whereas object detection is used to detect multiple types of
objects within one scene, instance segmentation is more focused on identifying
individual instances within one specific type of object (e.g., two cars).

By the definition of instance segmentation task, we known that any algorithm
which goal is to complete this task must do two things. First, the algorithm
must detect the location of the object, draw a bounding box srounding the
object and the outline of the object. Secondly, the algorithm must be able to
identify the class of the object resign inside the bouding box we draw from
previous step. The algorithm must also aware the number of instance in this
class at the second step.

From section \ref{sec:cnn}, we have discussed about the structure and building
block of a convolutional neural networks (CNNs). Throgh the discussion we also
state how CNNs able to classify object in image classification task. However,
by the definition of image classification task, the task assume the image have
exactly one object and the algorithm will classify the class for entire image
based on that one object. Therefore, we know that if we were to consider each
bouding box as it own image, we can utilize a CNN to identify the class of the
object within the bounding box. This is the main idea behind defferent
variation of R-CNN for instance segmentation and objectt detection task. 